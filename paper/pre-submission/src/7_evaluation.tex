\chapter{評価}
\label{evaluation}
本章では、提案システムに対する評価について述べる。

\section{評価内容}
まず、5.4節で示した要件を提案システムが満たしているかどうかを定性評価する。
次に、STRIDEを用いた脅威分析を行い、システムが実運用に耐えうるかどうかを評価する。
\section{システムの定性評価}
第5章で提案・設計したシステムが5.4節で示した要件を満たしているかどうか、以下でそれぞれ評価する。
\subsection{データのコントロール可能性}
データのコントロール可能性とは、利用者が自身の自動車から抽出したデータにアクセスし、コントロールできることである。
本システムでは、利用者の自動車から抽出されたデータは自動車から直接モバイルへと送信され、利用者の手元でウォレットに格納されて任意のタイミングで検証者に提示することができる。
利用者はデータがどこに所在するか、そしてどのようなデータが含まれるのかを認知することができるうえ、VPの提示という形で自身の自動車から抽出されたデータに対するコントロール権を獲得している。
そのため、データのコントロール可能性を満たしていると言える。
\subsection{データの真正性の担保}
データの真正性の担保とは、データを検証した者は誰でも、そのデータが偽造・改ざんされていないことを確認できることである。
本システムでは、データは自動車でVCsへと加工され、対応する利用者のウォレットアプリケーションへ格納された後にVPとして提示される。
提示されたVPに含まれるDIDsからDID documentを解決し、公開鍵を引いてVPを確認することで、当該データが適格な自動車から利用者に渡され、検証者に提示されたことおよび偽造・改ざんされていないことを検証できる。
そのため、データの真正性の担保も満たしていると言える。
\section{STRIDEによる脅威分析}
次に、STRIDEによる脅威分析を行う。
まず、STRIDEについて整理した後にシステムのデータフローを示し、それを元に脅威を洗い出した上で評価を行う。
\subsection{STRIDEについて}
STRIDEとは、Microsoft社が提唱した脅威分析のモデリング手法であり、システムに対する脅威を6つに分類する。
脅威分析モデルを構築し脅威のリスクを調べることで、セキュリティホールの検知などに役立てることができる。
STRIDEが分類する脅威のカテゴリは以下の通りである。
\begin{table}[h]
    \centering
    \caption{STRIDEにおける脅威のカテゴリ}
    \label{tab:stride}
    \begin{tabular}{ll}
        \hline
        \hline
        カテゴリ & 説明 \\
        \hline
        Spoofing(なりすまし) & 第三者が正規の主体を装う \\
        Tampering(改ざん) & データやシステムの偽造 \\
        Repudiation(否認) & 主体による行為の否定 \\
        Information Disclosure(情報漏えい) & 機密情報の不正な流出 \\
        Denial of Service(サービス妨害) & 正当な利用者がシステムを利用できない \\
        Elevation of Privilege(権限昇格) & 不正に上位の権限が取得される \\
        \hline
    \end{tabular}
\end{table}
\\
 なお、STRIDEには、DFD(次の節で言及)に含まれる要素全てを対象として脅威を洗い出すSTRIDE-per-Elementと、DFDの信頼境界上の脅威を洗い出すSTRIDE-per-Interactionが存在するが、本稿ではSTRIDE-per-Elementを採用する。
\subsection{システムのDFD}
STRIDEにおける脅威分析においては、まず対象システムのDFD(Data Flow Diagram)、すなわちデータフロー図を作成する。
DFDはシステムの構成要素や機能、そしてデータの流れを示した図であり、データストア、プロセス、外部の主体、データフローという4種類の要素から構成される。
本システムのDFDは下図のようになる。
\begin{figure}[h]
    \centering
    \includegraphics[width=150mm]{src/public/dfd.png}
    \caption{提案するシステムのデータフロー図}
\end{figure}
\subsection{脅威と評価}

%%% Local Variables:
%%% mode: japanese-latex
%%% TeX-master: "./thesis"
%%% End:
\section{本章のまとめ}