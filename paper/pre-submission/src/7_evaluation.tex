\chapter{評価}
\label{evaluation}
本章では、提案システムに対する評価について述べる。

\section{評価内容}
まず、5.4節で示した要件を提案システムが満たしているかどうかを定性評価する。
次に、STRIDEを用いた脅威分析を行い、システムが実運用に耐えうるかどうかを評価する。
\section{システムの定性評価}
第5章で提案・設計したシステムが5.4節で示した要件を満たしているかどうか、以下でそれぞれ評価する。
\subsection{データのコントロール可能性}
データのコントロール可能性とは、利用者が自身の自動車から抽出したデータにアクセスし、コントロールできることである。
本システムでは、利用者の自動車から抽出されたデータは自動車から直接モバイルへと送信され、利用者の手元でウォレットに格納されて任意のタイミングで検証者に提示することができる。
利用者はデータがどこに所在するか、そしてどのようなデータが含まれるのかを認知することができるうえ、VPの提示という形で自身の自動車から抽出されたデータに対するコントロール権を獲得している。
そのため、データのコントロール可能性を満たしていると言える。
\subsection{データの真正性の担保}
データの真正性の担保とは、データを検証した者は誰でも、そのデータが偽造・改ざんされていないことを確認できることである。
本システムでは、データは自動車でVCsへと加工され、対応する利用者のウォレットアプリケーションへ格納された後にVPとして提示される。
提示されたVPに含まれるDIDsからDID documentを解決し、公開鍵を引いてVPを確認することで、当該データが適格な自動車から利用者に渡され、検証者に提示されたことおよび偽造・改ざんされていないことを検証できる。
そのため、データの真正性の担保も満たしていると言える。
\section{STRIDEによる脅威分析}
次に、STRIDEによる脅威分析を行う。
まず、STRIDEについて整理した後にシステムのデータフローを示し、それを元に脅威を洗い出した上で評価を行う。
\subsection{STRIDEについて}
STRIDEとは、Microsoft社が提唱した脅威分析のモデリング手法であり、システムに対する脅威を6つに分類する。
脅威分析モデルを構築し脅威のリスクを調べることで、セキュリティホールの検知などに役立てることができる。
STRIDEが分類する脅威のカテゴリは以下の通りである。
\begin{table}[h]
    \centering
    \caption{STRIDEにおける脅威のカテゴリ}
    \label{tab:stride}
    \begin{tabular}{ll}
        \hline
        \hline
        カテゴリ & 説明 \\
        \hline
        Spoofing(なりすまし) & 第三者が正規の主体を装う \\
        Tampering(改ざん) & データやシステムの偽造 \\
        Repudiation(否認) & 主体による行為の否定 \\
        Information Disclosure(情報漏えい) & 機密情報の不正な流出 \\
        Denial of Service(サービス妨害) & 正当な利用者がシステムを利用できない \\
        Elevation of Privilege(権限昇格) & 不正に上位の権限が取得される \\
        \hline
    \end{tabular}
\end{table}
\\
 なお、STRIDEには、DFD(次の節で言及)に含まれる要素全てを対象として脅威を洗い出すSTRIDE-per-Elementと、DFDの信頼境界上の脅威を洗い出すSTRIDE-per-Interactionが存在するが、本稿ではSTRIDE-per-Elementを採用する。
\subsection{システムのDFD}
STRIDEにおける脅威分析においては、まず対象システムのDFD(Data Flow Diagram)、すなわちデータフロー図を作成する。
DFDはシステムの構成要素や機能、そしてデータの流れを示した図であり、データストア、プロセス、外部の主体、データフローという4種類の要素から構成される。
各要素は信頼できる要素ごとにグループ化され、信頼境界という境界線で区切られる。
なお、信頼境界はDFDにおいては点線で表現される。
本システムにおいては、信頼境界は3つに分かれ、それぞれ自動車、モバイル、検証者である。
また、STRIDEにおいては、脅威カテゴリとDFDの各要素との対応関係が定義されており、これに基づき各要素に対して検討するべき脅威カテゴリが決定される。
本システムのDFDは下図のようになる。
\begin{figure}[h]
    \centering
    \includegraphics[width=150mm]{src/public/dfd.png}
    \caption{提案するシステムのデータフロー図}
\end{figure}
\subsection{特定された脅威}
本システムでは、自動車、モバイル、検証者のそれぞれが信頼境界を形成している。
DFDを元に、それぞれの信頼境界内の各要素について脅威分析を行った。
脅威分析で特定された脅威は多くあるが、本研究の適用の範囲外のものも存在するため、より重要な脅威に絞って取り上げる。
なお、脅威分析の結果の詳細は付録Aの「STRIDEによる脅威分析の結果」に掲載したので参照されたい。
\subsection{自動車における脅威}
\begin{description}
    \item[\textbf{VCs発行プロセス}]
    VCs発行プロセスが改ざんされて不正に行われたり、正当な利用者になりすました第三者がVCsの発行を受けることを試みるような脅威である。
    この脅威により、VCsの正当性が意味をなさなくなったり、利用者を検証者が疑う必要などが出てきてしまう。
    対策としては、VCsを発行するプロセスを耐タンパ性のあるセキュアな領域で実施する、利用者の認証を適格に行うなどが考えられる。
    なお、利用者の認証に関しては、超広帯域通信(UWB: Ultra-Wide Band)および低消費電力通信(BLE: Bluetooth Low Energy)を活用した研究などがなされている。
    \item[\textbf{連携プロセス}]
    連携プロセスが改ざんされて不正に行われたり、正当な利用者になりすました第三者が連携をしようとしたりする脅威である。
    この脅威により、発行されたVCsが悪意ある第三者のもとに流れる可能性が生じ、さらに悪意ある第三者が正当な利用者のふりをして検証者にVPを提示することなどにつながってしまう。
    対策としては、利用者の認証を適格に行うことが考えられる。
    \item[\textbf{公開鍵情報}]
    公開鍵情報が外部VDRに格納されるまでの間に改ざんされる脅威である。
    この脅威により、検証者がVPを正当な手順で検証できないという問題が発生しうる。
    対策としては、安全が確立された通信経路を用いる、あるいは対処療法的に、検証者はVPを検証する前に取得した公開鍵の正当性を一回検証する、などが考えられる。
\end{description}
\subsection{モバイルにおける脅威}
\begin{description}
    \item[\textbf{VCs送受信プロセス}]
    VCs送受信プロセスが改ざんされて不正に行われたり、正当なウォレットアプリケーションになりすましたアプリケーションを操作する第三者がVCsの受信を受けたることを試みるような脅威である。
    この脅威により、正しくVPを構成できなくなるという問題が発生しうる。
    対策としては、連携アプリケーションとウォレットアプリケーションの連携を適格に行い、怪しい、認証していないようなアプリケーションと接続をしないようにすることなどが考えられる。
    \item[\textbf{連携プロセス}]
    自動車と同様の問題がモバイルでも発生しうる。
    \item[\textbf{公開鍵情報}]
    自動車と同様の問題がモバイルでも発生しうる。
\end{description}
\subsection{検証者における脅威}
\begin{description}
    \item[\textbf{検証者}]
    外部の主体である検証者に、悪意ある第三者がなりすます脅威である。
    この脅威により、利用者と検証者がVPを提示・検証することで達成したい目的を達成できないという問題が発生しうる。
    対策としては、検証前に検証者の認証を利用者が適格に行えるような仕組みを作成することなどが考えられる。
\end{description}

%%% Local Variables:
%%% mode: japanese-latex
%%% TeX-master: "./thesis"
%%% End:
\section{本章のまとめ}
本章では、提案するシステムに対し、5.4節で示した要件を満たしているかの定性評価と、STRIDEによる脅威分析という形で評価を行った。
次章では、本研究のまとめと課題、そして展望を示す。