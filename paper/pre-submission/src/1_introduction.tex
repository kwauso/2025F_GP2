\chapter{序論}
\label{introduction}

本章では本研究の背景,課題及び手法を提示し,本研究の概要を示す.

\section{はじめに}

インターネット技術の発展に伴い、あらゆるモノがネットワークに接続されるInternet of Thingsの概念が急速に広がっている。自動車もその例外ではなく、車両に通信機能を搭載することで、外部ネットワークと常時接続されたコネクティッドカーへと進化してきた。
とりわけ、テレマティクス技術の発展は、自動車の運転や利用状況に関する詳細なデータ収集を可能にしてきた。
利用ベース保険やカーシェアリングサービスにおける運転データの活用はその代表であり、車両データに基づく多様な価値提供が進んでいる。
しかし一方で、現在普及しているコネクティッドカーを基盤としたサービスは、データが自動車メーカーや特定のサービス事業者のクラウド内に閉じたサイロ構造を形成している。
そのため、利用者が自身の運転データを他の事業者へ持ち出したり、サービスを乗り換える際に運転履歴を引き継いだりすることは困難である。
また、利用者自身が自らのデータへアクセスできない、あるいは事業者は利用者が提示してきたデータの真正性が検証できないといった課題も存在する。
本研究では、このような課題に対し、Decentralized IdentifierおよびVerifiable Credentialsを活用した新たな運転歴データの管理・提示モデルを検討する。
これにより、利用者自身がデータに対するコントロールを保持しつつ、データの真正性を担保し、複数の事業者間で安全かつ信頼可能な運転歴データの再利用が可能となる仕組みの構築を目的とする。
\section{本論文の構成}

本論文における以降の構成は次の通りである.

~\ref{background}章では,背景を述べる.
~\ref{issue}章では,本研究における問題の定義と,解決するための要件の整理を行う.
~\ref{proposed}章では,本研究の提案手法を述べる.
~\ref{implementation}章では,~\ref{proposed}章で述べたシステムの実装について述べる.
~\ref{evaluation}章では,\ref{issue}章で求められた課題に対しての評価を行い,考察する.
~\ref{conclusion}章では,本研究のまとめと今後の課題についてまとめる.


%%% Local Variables:
%%% mode: japanese-latex
%%% TeX-master: "../thesis"
%%% End:

\cite{Bitcoin}
