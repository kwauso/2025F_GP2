\chapter{本研究における問題定義}
\label{issue}



%%% Local Variables:
%%% mode: japanese-latex
%%% TeX-master: "./thesis"
%%% End:

\section{既存サービスにおけるデータの流れ}
2.2節及び2.3節で紹介した既存サービスにおいて、基本的にコネクティッドカーから収集されたデータは自動車からサービスの事業者が接続しているクラウドへと流れ、クラウドで処理をされた後に事業者へと提供される。
トヨタの提供するT-Connectのように、モバイルアプリケーションへの配信などを通じて利用者にデータを提供するサービスも一部存在するが、基本的にデータは自動車とサービス事業者間を繋ぐネットワーク内にのみ存在していると言える。
\section{既存サービスの問題点}
前節で述べたデータの流れは、既存サービスのような利用者と事業者の間のみで完結するユースケースにおいては十分である。
しかし、特定の事業者とのみデータを共有する既存サービスは、ベンダーロックインを招きやすく、利用者にサービスの乗り換えという選択肢を与えない。
例えば、利用者が保険会社を乗り換えても利用ベース保険の実績を引き継ぐことができるかや、カーシェアリングサービスをまたいでも安全運転の履歴により割引を受けられるかはひとえに保険会社やカーシェアリングサービスの判断に委ねられている。
言い換えるならば、複数の事業者にまたがってデータを共有する際、既存サービスでは事業者やメーカー間の連携が不可欠であるという問題点が存在する。
また、このようなシステムでは、複数の事業者が結託して利用者に断りなくデータを共有する事態が発生する可能性も否定できない。
\\
 次項からは、この問題を構成している要因についてそれぞれ述べる。
\subsection{データのサイロ化}
前述したように、既存サービスにおいてはコネクティッドカーから収集されたデータは自動車と事業者が接続しているクラウドへと流れ、クラウドで処理をされた後に事業者へと提供される。
このようなシステムにおいては、利用者と事業者の間にデータが閉じ、サイロ化とでも言うべき状態が生まれる。
利用者が複数事業者を利用する場合、互いに独立したサイロが事業者の数だけ複数存在していることになり、これが複数の事業者間の連携を難しくしている要因の一つである。
\subsection{データのコントロール権の欠如}
また、サイロが構成されていると言っても、データはコネクティッドカーから抽出された後に事業者と接続したクラウドにて処理・蓄積されるため、利用者がデータにアクセスしたり、データをコントロールしたりすることは不可能に近い。
仮に利用者が自分の自動車から抽出したデータにアクセスし、それをコントロールすることが可能であれば、そのデータを事業者をまたいで提示する可能性が出てくるが、そのようなシステムは現在存在しない。
\subsection{データの真正性の担保}
仮に利用者がデータを提示してきたとしても、それを事業者が信頼できるとは限らない。
例えば、利用者がデータを偽造・改ざんしたり、他人の自動車から抽出したデータを提示してきたりする可能性を否定できないからである。
そのため、既存サービスではデータの真正性を担保するために信頼できる自動車メーカーが製造したコネクティッドカーから信頼できる通信経路を経由してクラウドにデータを送信している。
\subsection{各事業者のポリシーの相違}
さらに、これまでに述べた要因が取り除かれたとしても、各事業者のポリシーの相違によってデータを共有することができない場合がある。
各事業者はそれぞれ異なるポリシーを持っており、判断に用いられるデータは同じでも、例えば保険会社では総合した安全運転の実績を重視する一方、カーシェアリングサービスでは急な加減速の有無を重視するなど、ポリシーによってデータの評価基準が異なる。
そのため、ポリシーによって処理される前の状態のデータを共有する必要が出てくる可能性がある。
\section{本章のまとめ}