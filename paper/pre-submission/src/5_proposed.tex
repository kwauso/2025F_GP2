\chapter{提案手法}
\label{proposed}

本章では、第3章で提示した問題を解決するために、4.5節で提示する前提条件のもとで利用者のデータへのコントール可能性とデータの真正性の担保を両立する手法およびシステムを提案する。




%%% Local Variables:
%%% mode: japanese-latex
%%% TeX-master: "../bthesis"
%%% End:

\section{提案手法の概要}
第4章で示したように、既存システムではデータが自動車と事業者の間に閉じることでデータがサイロ化し、利用者にとってサービスの乗り換えが難しくなり、また自身の自動車から抽出したデータにアクセス・コントロールすることができないためデータがどのように扱われているか感知できないという問題が存在する。
また、事業者から見てもポリシーの差異によりデータを元に判定・スコアリングされた利用者をそのまま受け入れることはできず、さらにデータそのものを提示しても自動車から直に抽出した物でない以上利用者による改ざん・なりすましのリスクがつきまとうため受け入れることができないことも示した。
\\
 以上の問題を解決するために、提案手法ではデータの流れに新たに利用者を加え、自動車から抽出したデータをVCsで表現した上で利用者に渡す。
利用者はVCsを受け取った後、ウォレットアプリケーションに格納し、望むタイミングでVCsないしはVPとして検証者に提示することができる。
検証者はVCsないしはVPに含まれる公開鍵情報をもとに検証を行い、データの真正性を検証できる。
VCsの特性により、検証者は当該VCsが適切な自動車から発行された物であることが確認でき、利用者も自身の自動車から抽出したデータに対するアクセスおよびコントロールを獲得できる可能性がある。
本研究ではこの手法が実現可能であることを示すために、提案手法に沿ったシステムを設計した。
\section{全体構造}
本節では、提案するシステムのアーキテクチャについて示した後、データおよび処理がどのように流れるかを示し、さらに具体的なコンポーネントとしての自動車アプリケーション、モバイルアプリケーション、検証者アプリケーションについて説明する。
\subsection{システムアーキテクチャ}
提案するシステムの論理アーキテクチャ図は以下の通りである。
\begin{figure}[h]
    \centering
    \includegraphics[width=150mm]{src/public/systemarc_whole.png}
    \caption{提案するシステムの論理アーキテクチャ}
\end{figure}
本システムは示すように、自動車側、モバイル側、検証者側の3つのコンポーネントから構成される。
 まず、自動車とモバイルアプリケーションを接続し、信頼できる通信路を確立する。
自動車では、各センサから取得された情報をデータ辞書に格納し、VCに加工した上でモバイルアプリケーションへ送信する。
モバイルでは取得したVCをウォレットアプリケーションに蓄積し、適宜検証者に提示する。
以上が、システムの一連の流れである。
\subsection{処理の流れ}
\subsection{自動車アプリケーション}
自動車アプリケーションは、モバイルアプリケーションとの間の信頼できる通信経路の確立と自動車内の各センサから取得した情報をVCに加工し送信するという2つの機能を持つ。
通信経路の確立は、公開鍵基盤に基づいて行われ、自動車とモバイルアプリケーションそれぞれが持つ鍵ペアを用いて行われる。
また、自動車はメーカーから割り振られた一意の自動車DIDと、VCに添付するための複数のデータDIDを持つ。
自動車が発行するVCに、自動車DIDとデータDIDが紐づくことを示すVCを添付することにより、データDIDの出所を担保する。
なお、データをVCに加工するときに用いる鍵ペアは、モバイルアプリケーションとの間の信頼できる通信経路を確立するときに用いたものと同じものを用いることを想定している。
\subsection{モバイルアプリケーション}
モバイルアプリケーションは、自動車との間の信頼できる通信経路の確立と自動車から送信されたVCをウォレットアプリケーションに蓄積し、検証者に提示するという2つの機能を持つ。
なお、通信経路の確立については、前項で述べたとおりである。
また、受け取ったVCをウォレットアプリケーションに送信する役割を持ち、ウォレットアプリケーションはそれをVPに加工し、必要なときに検証者に提示する役割を持つ。
\subsection{検証者アプリケーション}
検証者アプリケーションは、実際の事業者により多様に異なる形をとると想定されるが、基本的な役割としては、送信されたVPを確認してその有効性及び真正性を検証することである。
すなわち、必要な機能としては受け取ったVPをパースし、記載されるDIDを解決した上で、含まれるデータを取り出すことである。
\section{データ辞書}
データ辞書は、システムの共通言語としてのデータモデルであり、自動車側、モバイル側、検証者側の3つのコンポーネントが共通のデータモデルを用いて通信を行うことを目的としている。
\subsection{基本設計}
基本設計として、データ辞書は抽象された自動車情報をJSON形式で含み、一定時間の間の状態を記述する。
自動車から取得される情報は、車両やセンサごとに種類や精度にばらつきがあり、それら全てを網羅したデータ構成を用意することは非常に難しい。
また、車両の性質による特殊な情報や事業者のポリシーに依存した情報もあり、これらも全て含むことは非常に困難であるばかりかデータ量を増大させシステムの肥大化をも招く。
そのため、検証に必要であると想定できるデータセットを定義することで事業者ごとのポリシーの差を吸収し、システムの基盤として利用していく必要がある。
なお、データは正規化されていることを想定している。
例えば、センサのパルス数ではなく速度として算出する、電圧ではなくon/offの状態として算出するなどである。
\subsection{データセット構造}
データセットとして、以下がデータ辞書に含まれる。各データは全て自動車の状態である。
\begin{itemize}
    \item 速度
    \item 加速度
    \item 角速度
    \item 回転数
    \item 位置情報
    \item ライト状態
    \item ブレーキ状態
    \item 車両イベント
    \item スロットル開度
    \item ステアリング角度
\end{itemize}
\subsection{データ辞書の論理構成}
/*(図を載せます)*/
\newline
データ辞書の論理構成は以下である。
データの収集、加工、提供、検証を考慮したアーキテクチャを検討し、この一連の処理の構成をモデル化した。
\subsection{データ辞書の物理構成}
/*(図を載せます)*/
\newline
データ辞書の物理構成は以下である。
\section{システムの満たすべき要件}
システムは満たすべき要件として以下が挙げられる。
\subsection{データのコントロール可能性}
利用者が自身の自動車から抽出したデータにアクセスし、コントロールできることである。
ここでのコントロールとは、データがどこに所在してどのような内容が含まれるか利用者が把握しており、自由に任意の事業者に提示することができることであり、利用者が自由にデータを改変したり移転したりすることができることではない。
\subsection{データの真正性の担保}
データを検証した者は誰でも、そのデータが偽造・改ざんされていないことを確認できることである。
具体的には、事業者が利用者から提示されたVPを検証したとき、記載されたDIDを解決することで含まれるデータが当該自動車から取得されたものであり、利用者によって偽造されたり改ざんされたりしていないことを検証できることを指す。
\section{前提条件}
本提案手法は以下の条件を前提としている。
\subsection{自動車は純正であること}
自動車はメーカーが製造したままの状態であり、改造などを施されていない状態(純正)であることを想定している。
例えば、自動車愛好家の間で行われるようなカスタム・チューニングを施された自動車などであれば、この条件を満たさない。
センサーなどが改変されることは利用者がデータを改ざんしたことと同義であり、要件を満たせなくなるため、このような自動車は本提案手法の対象外とした。
\subsection{自動車メーカーは信頼に足ること}
自動車メーカーが自動車の製造時に不正な行為を行わないことを想定している。
例えば、自動車メーカーがセンサーを改造し実速度よりも表示速度を大幅に下げてい場合、この条件を満たさない。
事業者が自動車から抽出したデータを信頼しているのは自動車メーカーを信頼しているためであることは前述のとおりであるが、自動車メーカー自体が不正な行為を行うとその信頼が損なわれるため、この条件を満たさない。
\subsection{自動車はコネクティッドカーであること}
本提案手法が対象とするのはコネクティッドカーである。
ECUやTCUを持たない自動車はそもそもモバイルアプリケーションとの間に通信経路を確立したり、データを抽出したりすることができないため、そのような自動車は本提案手法の対象外とした。
\section{本章のまとめ}