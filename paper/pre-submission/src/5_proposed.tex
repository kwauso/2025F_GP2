\chapter{提案手法}
\label{proposed}

本章では、第4章で提示した問題を解決するために、5.5節で提示する前提条件のもとで利用者のデータへのコントール可能性とデータの真正性の担保を両立する手法およびシステムを提案する。


%%% Local Variables:
%%% mode: japanese-latex
%%% TeX-master: "../bthesis"
%%% End:

\section{提案手法の概要}
第4章で示したように、既存システムではデータが自動車と事業者の間に閉じることでデータがサイロ化し、利用者にとってサービスの乗り換えが難しくなり、また自身の自動車から抽出したデータにアクセス・コントロールすることができないためデータがどのように扱われているか感知できないという問題が存在する。
また、事業者から見てもポリシーの差異によりデータを元に判定・スコアリングされた利用者をそのまま受け入れることはできず、さらにデータそのものを提示しても自動車から直に抽出した物でない以上利用者による改ざん・なりすましのリスクがつきまとうため受け入れることができないことも示した。
\\
 以上の問題を解決するために、提案手法ではデータの流れに新たに利用者を加え、自動車から抽出したデータをVCsで表現した上で利用者に渡す。
利用者はVCsを受け取った後、ウォレットアプリケーションに格納し、望むタイミングでVCsないしはVPとして検証者に提示することができる。
検証者はVCsないしはVPに含まれる公開鍵情報をもとに検証を行い、データの真正性を検証できる。
VCsの特性により、検証者は当該VCsが適切な自動車から発行された物であることが確認でき、利用者も自身の自動車から抽出したデータに対するアクセスおよびコントロールを獲得できる可能性がある。
本研究ではこの手法が実現可能であることを示すために、提案手法に沿ったシステムを設計した。
\section{全体構造}
本節では、まず提案するシステムのアーキテクチャについて示した後、データおよび処理がどのように流れるかを示し、さらに具体的なコンポーネントとしての自動車、モバイル、検証者のそれぞれが持つアプリケーションについて説明する。
\subsection{システムアーキテクチャ}
提案するシステムの論理アーキテクチャ図は以下の通りである。
\begin{figure}[h]
    \centering
    \includegraphics[width=150mm]{src/public/systemarc_whole.png}
    \caption{提案するシステムの論理アーキテクチャ}
\end{figure}
\\
 本システムは上図が示すように、自動車、モバイル、検証者の3つのコンポーネントから構成される。
\\
 自動車は通常コネクティッドカーが持つ機能・コンポーネント(各種センサ、ECUや TCUなどの車両制御ユニット、鍵ペア、識別子など)に加え、データを処理するための処理アプリケーションを持つ。
処理アプリケーションの持つ機能は、モバイルとの連携、自動車から取得したデータを元にVCsを発行、VCsの送信、状態管理である。
モバイルは通常モバイルが持つ機能・コンポーネント(鍵ペア、識別子など)に加え、自動車と連携するための連携アプリケーションとVCsを保持・提示するためのウォレットアプリケーションを持つ。
連携アプリケーションの持つ機能は自動車との連携、VCsの受け取り、ウォレットアプリケーションへの送信、状態管理であり、ウォレットアプリケーションの持つ機能はVCsの保持、VPの構成、VCsおよびVPの提示である。
検証者は検証アプリケーションを持つ複数の事業者である。
検証アプリケーションは提示されたVPから公開鍵を解決し、真正性の検証を行う機能を持つ。
以下は、各コンポーネントが持つアプリケーションとその機能をまとめた表である。
\begin{table}[h]
    \centering
    \caption{システムの各コンポーネントが持つアプリケーションとその機能}
    \label{tab:component}
    \begin{tabular}{llp{7.5cm}}
        \hline
        \hline
        コンポーネント & アプリケーション & 機能 \\
        \hline
        自動車 & 処理アプリケーション & モバイルとの連携、VCsの発行、VCsの送信、状態管理 \\
        モバイル & 連携アプリケーション & 自動車との連携、VCsの受け取り、ウォレットアプリケーションへの送信、状態管理 \\
        & ウォレットアプリケーション & VCsの保持、VPの構成、VCsおよびVPの提示 \\
        検証者 & 検証アプリケーション & 公開鍵の解決、真正性の検証 \\
        \hline
    \end{tabular}
\end{table}
\subsection{データおよび処理の流れ}
提案するシステムのシーケンス図は以下の通りである。
\begin{figure}[h]
    \centering
    \includegraphics[width=150mm]{src/public/systemsequence.png}
    \caption{提案するシステムのシーケンス図}
\end{figure}
システムは、処理の段階に応じて3つのフェーズに分類される。
それぞれ連携フェーズ、データ処理フェーズ、検証フェーズであり、関係するコンポーネント・アプリケーションが異なる。
\\
 まず、連携フェーズにおいて、VCsを発行するべく自動車の処理アプリケーションとモバイルの連携アプリケーションが連携する。
連携の際、処理アプリケーションは自動車内の、連携アプリケーションはモバイル内の鍵ペアをそれぞれ用い、チャンレンジレスポンス認証を行った上で通信経路を確立する。
通信経路が確立すると、それぞれのコンポーネント内で連携の状態を管理しているデータベースなどと通信し、連携状態を同期する。
\\
 自動車とモバイルの連携が完了すると、実際にVCsを発行することができるため、処理フェーズへ移行する。
自動車が各種センサからデータを取得し、EUCやTCUなどの車両制御ユニットを介して処理アプリケーションへとデータが送られる。
処理アプリケーションでは同期されているVCs発行状態を参照しながら、連携時に自動車から取得した鍵ペアを用いてVCsを作成し、連携フェーズで確立した通信路を用いてモバイルにVCsを逐次送信する。
モバイルでは連携アプリケーションがVCsを逐次受け取り、ウォレットアプリケーションへと送信する。
ウォレットアプリケーションでは受け取ったVCsを格納し、保持する。
\\
 検証者が検証情報の提示を求める状態が発生すると検証フェーズへと移行する。
このような事態になる要因としては、利用者が新規サービスを申し込む、既存サービスを乗り換える、などが考えられる。
利用者はモバイル内の鍵ペアを用いてウォレットアプリケーションに格納されたVCsからVPを構成し、検証者へと提示する。
なお、実際のデータは複数回にわたりVCsへ処理されることが想定され、また車両のDIDsの真正性を担保するためのVCsも必要であることから利用者が検証者へ提示するのは必然的にVPになる。
検証アプリケーションはVPを受け取ったら、含まれるDIDsからDID documentを解決し、自動車と利用者の公開鍵情報を取得する。
取得した公開鍵情報を元にVPを検証し、検証結果を利用者へと返す。
\\
 以上が一連の処理の流れである。
\subsection{処理アプリケーション}
処理アプリケーションは、モバイルの連携アプリケーションとの間の通信経路の確立と、自動車内の各種センサから取得した情報をVCsに加工し、連携アプリケーションに送信するという2つの機能を持つ。
通信経路の確立は、公開鍵基盤に基づいて行われ、自動車とモバイルアプリケーションそれぞれが持つ鍵ペアを用いて行われる。
また、自動車はメーカーから割り振られた一意の自動車DIDsと、VCsに添付するための複数のデータDIDsを持つ。
VCsを発行する際に、自動車DIDsとデータDIDsが紐づくことを示すVCsを添付することにより、データDIDsの出所を担保する。
なお、この紐付きを示すVCsは、連携時に一度だけ添付すれば十分である。
また、データをVCsに加工するときに用いる鍵ペアは、連携アプリケーションとの間の通信経路を確立するときに用いたものと同じものを用いる。
\subsection{連携アプリケーション}
連携アプリケーションは、自動車の処理アプリケーションとの間の通信経路の確立と、処理アプリケーションから送信されたVCsをウォレットアプリケーションに送信するという2つの機能を持つ。
なお、通信経路の確立については、前項で述べたとおりである。
連携アプリケーションは、連携時にモバイルの鍵ペアを用いる。
\subsection{ウォレットアプリケーション}
ウォレットアプリケーションは、同じモバイル内の連携アプリケーションから送信されたVCsを格納・保持することと、VCsからVPを構成した上で検証者に提示するという2つの機能を持つ。
ウォレットアプリケーションは、VPの構成時にモバイルの鍵ペアを用いる。
\subsection{検証者アプリケーション}
検証者アプリケーションは、実際の事業者により多様に異なる実装をとると想定されるが、基本的な役割としては、送信されたVPを確認してその有効性及び真正性を検証することである。
すなわち、必要な機能としては受け取ったVPをパースした後含まれるDIDsからDID documentを解決し、自動車および利用者の公開鍵情報を取得してVPが適格に発行され、改ざんされておらずかつ失効していないことを検証することである。
\section{データモデル}
システム内で各コンポーネントとアプリケーションが連携するには、連携の基盤としてのデータモデルが必要である。
本節では、具体的にどのようなデータがデータモデル含まれ、使用されるかについて述べる。
\subsection{基本設計}
基本設計として、データモデルには抽象化された自動車情報が含まれ、一定時間の間の状態を記述する。
前提として、自動車から取得される情報は、車両やセンサごとに種類や精度にばらつきがあり、それら全てを網羅したデータ構成を用意することは非常に難しい。
また、車両の性質による特殊な情報やメーカーのポリシーに依存した情報もあり、これらも全て含むことは非常に困難であるばかりかデータ量を増大させ、システムの肥大化をも招いてしまう。
そのため、検証に必要であると想定されるデータセットを定義することでこれらのばらつきや差を吸収し、システムの基盤として利用していく必要がある。
\\
 なお、データは正規化されていることを想定している。
例えば、センサのパルス数ではなく速度として算出されたものを使用する、電圧ではなくon/offの状態として算出されたものを使用するなどである。
\subsection{データの構造}
データセットとして、本研究では速度、加速度、回転角速度、ステアリング角度をデータモデルに含むこととした。
各データは全て自動車の状態である。
データモデルの策定に際しては、PHYDタイプのUBI保険や道路交通安全に関連する国際規格・国際標準を参照した。
以下は、データモデルに含まれるデータが何を検出するためか、および実際にどのようなセンサから取得できるかを示した表である。
\begin{table}[h]
    \centering
    \caption{システムで使用されるデータとセンサ}
    \label{tab:dataset}
    \begin{tabular}{llp{7.5cm}}
        \hline
        \hline
        データ & 目的 & センサ \\
        \hline
        速度 & 速度超過率の検知 & 車速センサ \\
        加速度 & 急加速・急減速および加速度変化率の検知 & 加速度センサ \\
        回転各速度 & 急旋回の検知 & ジャイロセンサ \\
        ステアリング角度 & 急旋回の検知 & 舵角センサ \\
        \hline
    \end{tabular}
\end{table}
\section{システムの満たすべき要件}
提案するシステムは、以下の要件を満たすことが望ましい。
\subsection{データのコントロール可能性}
利用者が自身の自動車から抽出したデータにアクセスし、コントロールできること。
ここで言うコントロールとは、データがどこに所在しており、かつどのような内容が含まれるか利用者自身が把握しており、また任意の事業者に対してデータを提示することが可能なことを指すものであり、利用者が自由にデータを改変したり移転したりすることができることを指すものではない。
\subsection{データの真正性の担保}
データを検証した者は誰でも、そのデータが偽造・改ざんされていないことを確認できること。
具体的には、検証者が利用者から提示されたデータを検証したとき、データが適格かつ望ましい自動車から取得されたものであり、利用者が他者になりすまして他者の自動車から取得したデータを提示してきていないことや、データを偽造あるいは改ざんした上で提示してきていないことを検証できることを指す。
\section{前提条件}
本提案手法は以下の前提条件を持つ。
\begin{description}
    \item[\textbf{自動車は純正である}]
    自動車はメーカーが製造したままの状態であり、改造などを施されていない状態(純正)であること。
    例えば、自動車愛好家の間で行われるようなカスタム・チューニングを施された自動車などであれば、この条件を満たさない。
    ここで言う「改造」とは、メーカーが自動車を設計した段階で利用者が自動車はこのように使用すると想定した方法から著しく逸脱して使用されていないことを指し、利用者が世間と同程度のレベルでの使用をしていれば満たされるであろう条件である。
    センサやECU, TCUなどが改変されることはデータが改ざんされることと同義と言え、前述した要件を満たせなくなるため、このような改造された自動車は本提案手法の適用の対象外とした。
    \item[\textbf{自動車メーカーは不正を行わない}]
    自動車メーカーが自動車の製造時に不正な行為を行わないこと。
    例えば、自動車メーカーが実馬力よりも低い数値を公表したりしていた場合、不正行為にあたりこの条件を満たさない。
    検証に際し、事業者が自動車から抽出したデータが真正性を持つと考えているのは自動車メーカーが不正を行わないと信じているためであることは前述のとおりであるが、自動車メーカー自体が不正な行為を行うとその信用が損なわれてしまう。
    そのため、このような事例は本提案手法の検討対象外とした。
    \item[\textbf{自動車はコネクティッドカーである}]
    提案手法の対象となる自動車はコネクティッドカーであること。
    電子制御を持たない自動車はECUやTCUを持たず、結果当然として鍵ペアなども持てないため、モバイルとの間に通信経路を確立したり、データを送信することはおろか、データを抽出することすら困難な場合が多い。
    その場合、提案手法が一切適用できなくなってしまうため、このような自動車は本提案手法の適用の対象外とした。
\end{description}
\section{本章のまとめ}
本章では、第4章で提示した問題を解決するための提案手法を示し、具体的なシステムの構成・データモデル・要件を設計した上で前提条件にも言及した。
次章では、提案したシステムを実際に実装し、動作させる。