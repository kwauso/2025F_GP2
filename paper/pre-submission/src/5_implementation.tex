\chapter{実装}
\label{implementation}

本章では、第4章で提案したシステムの実装について述べる。




%%% Local Variables:
%%% mode: japanese-latex
%%% TeX-master: "../bthesis"
%%% End:


\section{実装の概要}
前章で提案した手法を実装し、実際に動作するシステムを構築する。
実装するのは自動車アプリケーション、モバイルアプリケーション、検証者アプリケーションの3つであるが、自動車アプリケーションに関しては実際の自動車で動作するアプリケーションを開発することは労力が大きいため、Raspberry Piを用いたエミュレータと実装した。
\subsection{機能要件}
それぞれのアプリケーションの機能要件については、以下とする。
\begin{enumerate}
    \item 自動車アプリケーション \newline
        ・基盤内の鍵ペアを用い、モバイルアプリケーションとの間の信頼できる通信経路の確立 \newline
        ・自動車内の各センサから取得した情報をデータ辞書に基づいてVCに加工し、モバイルアプリケーションへ送信
    \item モバイルアプリケーション \newline
        ・自動車との間の信頼できる通信経路の確立 \newline
        ・自動車から送信されたVCをウォレットアプリケーションに蓄積し、VPに加工 \newline
        ・VPを検証者に提示
    \item 検証者アプリケーション \newline
        ・受信したVPを確認してその有効性及び真正性を検証し、データを取り出す
\end{enumerate}
\subsection{システム構成}
/*(図を載せます)*/
\newline
全体のシステム構成は以下である。
\subsection{全体の開発環境}
実装にあたり、開発環境は以下を用いた。
\section{データモデル}
データモデルは第4章で述べたとおり、JSON形式で表現されたデータ辞書を用いる。
\section{自動車アプリ}
実装にあたり、自動車アプリケーションはRaspberry Piを用いたエミュレータとして実装した。
\section{モバイルアプリ}
実装にあたり、モバイルアプリケーションはFlutterを用いて実装した。
\section{検証者アプリ}
実装にあたり、検証者アプリケーションはTypescriptを用いて実装した。
\section{VC生成・署名の方式}
/*(図を載せます)*/
