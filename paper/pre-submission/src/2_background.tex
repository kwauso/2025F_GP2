\chapter{背景}
\label{background}

本章では本研究の背景について述べる.


\if0
\begin{figure}[h]
    \begin{center}
        \includegraphics[scale=0.4]{./img/hashrate.png}
        \caption{2017年1月のハッシュレート分布 出典:Blockchain.info\cite{bitcoinhashrate}}
        \label{img:hashrate}
    \end{center}
\end{figure}
\fi
\section{自動車の情報化}
インターネットの発展に伴い、いまやパソコンだけではなく、あらゆるものがインターネットと接続することが当たり前になった。
このような技術は"Internet of Things"の略称であるIoTと呼ばれ、社会のデジタル化に欠かせない技術となっている。
自動車も例外ではなく、インターネットと自動車を接続することで自動車自体を情報化し、道路交通情報をシステム化することができる。
道路交通に関する総合的な情報通信システムは高度道路交通システムと称され、"Intelligent Transport Systems"の略称であるITSと呼ばれる。
自動車の情報化は社会全体の利益に繋がり、その必要性は高い。
同時に、自動車と情報通信に関連する分野で、情報通信社会を支援する役割も期待されている。

\section{コネクティッドカーとテレマティクス}
本節では、コネクティッドカーとそれを支える技術であるテレマティクスについて概説し、その活用事例及び発展と展望を紹介する。
\subsection{コネクティッドカーとテレマティクスの概要}
コネクティッドカーとは、外部ネットワークと接続している自動車のことである。
自動車に通信システムを搭載することで、リアルタイムに外部とデータのやり取りを行うことができ、これをテレマティクスと呼ぶ。
なお、テレマティクスはIoTの一種であり、「テレマティクス(telematics)」という語は「テレコミュニケーション(telecommunication)」と「インフォマティクス(informatics)」を組み合わせた造語である。
コネクティッドカーは、テレマティクスによって外部ネットワークに接続している自動車そのものである。
テレマティクスに対応したコネクティッドカーにはTCU(テレマティクス制御ユニット)と呼ばれる部品が搭載され、TCUは自動車のECU(電子制御ユニット)と接続し、自動車のデータを用いて外部ネットワークとの通信を行う。
また、コネクティッドカーでは「Over The Air(OTA)」という技術を用いることで車載ソフトウェアの更新を無線ネットワーク経由で行うことができる。
\subsection{テレマティクスの活用事例}
テレマティクスを用いることで、様々なシステムが実現している。
例えば、自動車の事故や事件などの非常事態が発生した際に、警察や救急、メーカーや保険会社などに緊急で通報する緊急通報システムや車両が盗難された時に位置情報を元に追跡できる車両盗難追跡システムなどである。
また、テレマティクスを用いるサービスも多く存在し、代表的なものとして利用ベース保険が挙げられる。利用ベース保険の詳細については後述する。
\subsection{コネクティッドカーの発展と展望}
コネクティッドカーは車両の通信機能によりリアルタイムで外部ネットワークとのデータのやり取りを行い、安全性と利便性、そしてシステム・サービス間の連携を大きく向上させてきた。
テレマティクスを用いるシステム・サービスの拡大に伴い、自動車が「移動するデジタル端末」として機能しており、今後は自動運転やEV、そしてMaaS(Mobility as a Service)との統合が進み、車両データを基盤とした新たなエコシステムの中心的役割を担うことも期待される。
コネクティッドカーはテレマティクスを用いるシステムの発展とともに新車に占める台数が年々増加しており、2030年には新規に出荷される乗用車のうち95\%以上を占めると予想されている。

\section{運転歴データを用いるサービス}
本節では、実際にコネクティッドカーから抽出したデータを利活用するサービスについて紹介する。
なお、自動車からデータを抽出する方法としてはOBD-IIコネクタやスマートフォン、EDR(Event Data Recorder。車載ブラックボックスとも呼ばれる)などが存在する。
\subsection{利用ベース保険(UBI)}
利用ベース保険(UBI)はテレマティクス保険とも呼ばれ、自動車から収集した運転データを基に保険料を算出する保険である。
従来の自動車保険は大数の法則に基づき、過去の膨大な統計データを元に保険料を算出していたが、利用ベース保険ではより詳細な、運転手ごとにカスタマイズされた保険料を算出することができる。
例えば、責任感を持ち、安全な運転志向を持つドライバーは大きな割引を受けることができる一方、危険な運転志向を持つドライバーに対しては保険料を高く設定することができ、これにより保険会社のリスクを低減することができる。
なお、保険料の算出方法として主流なものはPAYD (Pay As You Drive)とPHYD (Pay How You Drive) であり、前者は走行距離に連動し、後者は運転の仕方に連動するものである。
運転の仕方には、平均速度、加速や減速の度合い、運転する場所、運転する時間などが含まれる。
利用ベース保険は特に北米市場で高い需要があり、2030年までには世界市場で11.34\%の年平均成長率で拡大すると予想されている。
\subsection{カーシェアリングサービス}
\subsection{レンタカーサービス}

\section{Verifiable Credentials}
\subsection{Verifiable Credentialsの概要}
\subsection{選択的開示}
