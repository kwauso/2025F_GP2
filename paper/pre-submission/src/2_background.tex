\chapter{背景}
\label{background}

本章では本研究の背景について述べる。

\section{自動車の情報化}
インターネットの発展に伴い、コンピュータだけではなく、あらゆるものがネットワークで接続しあうことがもはや当たり前となった。
このような、モノとモノがネットワークを介して繋がる技術は``Internet of Things''の略称であるIoTと呼ばれ、社会のデジタル化に欠かせない技術となっている。
自動車も例外ではなく、ネットワークと自動車を接続することで自動車自体を情報化し、道路交通情報をシステム化することができる。
道路交通に関する総合的な情報通信システムは高度道路交通システムと称され、``Intelligent Transport Systems''の略称であるITSと呼ばれる。
自動車の情報化は社会全体の利益に繋がり、その必要性は高い。
例えば、複数の自動車からデータを集めて分析することで、渋滞の緩和に役立てようという試みはその一例である。
自動車の自動化は、自動車と情報通信に関連する分野に貢献することで、情報通信社会を支援する役割も期待されている。

\section{コネクティッドカーとテレマティクス}
本節では、コネクティッドカーとそれを支える技術であるテレマティクスについて概説し、その活用事例及び発展の展望を紹介する。
\subsection{コネクティッドカーとテレマティクスの概要}
コネクティッドカーとは、外部ネットワークと接続している自動車であると定義される。
自動車に通信システムを搭載することで、リアルタイムに外部とデータのやり取りを行うことができ、このように移動体に通信手段を搭載し、外部と情報をやり取りする技術をテレマティクスと呼ぶ。
なお、テレマティクスはIoTの一種であり、「テレマティクス(telematics)」という語は「テレコミュニケーション(telecommunication)」と「インフォマティクス(informatics)」を組み合わせた造語である。
コネクティッドカーは、テレマティクスによって外部ネットワークに接続している自動車そのものである。
テレマティクスに対応したコネクティッドカーにはTCU(Telematics Control Unit = テレマティクス制御ユニット)と呼ばれる部品が搭載され、TCUは自動車のECU(Electronic Control Unit = 電子制御ユニット)と接続し、外部ネットワークとの通信を行う。
また、コネクティッドカーでは``Over The Air(OTA)''という技術を用いることで車載ソフトウェアの更新を無線ネットワーク経由で行うことができる。
\subsection{テレマティクスの活用事例}
テレマティクスを用いることで、様々なシステムが実現している。
例えば、自動車が関係する事故や事件などの非常事態が発生した際に、警察や救急、メーカーや保険会社などに緊急で通報する緊急通報システムや、車両が盗難された時に位置情報を元に追跡が可能な車両盗難追跡システムなどである。
また、テレマティクスを用いるサービスも多く存在し、代表的なものとして利用ベース保険が挙げられる。利用ベース保険の詳細については後述する。
\subsection{コネクティッドカーの発展と展望}
コネクティッドカーは車両の通信機能によりリアルタイムで外部ネットワークとのデータのやり取りを行い、安全性と利便性、そしてシステム・サービス間の連携を大きく向上させてきた。
テレマティクスを用いるシステムやサービスの利用拡大に伴い、自動車が「移動するデジタル端末」として機能しており、今後は自動運転やEV、そしてMaaS(Mobility as a Service)との統合が進み、車両データを基盤とした新たなエコシステムの中心的役割を担うことも期待される。
コネクティッドカーはテレマティクスを用いるシステムの発展とともに新車に占める台数が年々増加しており、2030年には新規に出荷される乗用車のうち95\%以上を占めると予想されている\cite{fuji-keizai-connected-car-2021}。

\section{車両データを用いるサービス}
実際にコネクティッドカーから抽出したデータを利活用するサービスは多く存在するが、本節ではその中でも特に運転歴に関する車両データを用いるサービスについて紹介する。
運転歴に関する、とは速度や加速度、位置情報や時間、あるいはスロットル開度やステアリング角度など、自動車の動きの中でドライバーの「操作する」という動きと直接繋がって変化するデータであるとここでは定義する。
なお、自動車からデータを抽出する方法としてはOBD-IIコネクタやスマートフォン、EDR(Event Data Recorder。車載ブラックボックスとも呼ばれる)などが存在する。
\subsection{利用ベース保険(UBI)}
利用ベース保険(UBI)はテレマティクス保険とも呼ばれ、自動車から収集した運転データを基に保険料を算出する保険である。
従来の自動車保険は大数の法則に基づき、過去の膨大な統計データを元に保険料を算出していたが、利用ベース保険ではより詳細な、運転手ごとにカスタマイズされた保険料を算出することができる。
例えば、責任感を持ち、安全な運転志向を持つドライバーは大きな割引を受けることができる一方、危険な運転志向を持つドライバーに対しては保険料を高く設定することができ、これにより保険会社のリスクを低減することができる。
なお、保険料の算出方法として主流なものはPAYD (Pay As You Drive)とPHYD (Pay How You Drive) であり、前者は走行距離に連動し、後者は運転の仕方に連動するものである。
運転の仕方には、平均速度、加速や減速の度合い、運転する場所、運転する時間などが含まれる。
利用ベース保険は特に北米市場で高い需要があり、2030年までには世界市場で11.34\%の年平均成長率で拡大すると予測されている\cite{fitch-us-auto-2024}\cite{researchstation-ubi-2027}。
\subsection{カーシェアリングサービス}
カーシェアリングサービスは、登録された自動車を会員が共同で使用するサービスである。
登録される自動車の所有者が個人であるか法人であるかや、登録される自動車の種類によって異なる様々なサービスが存在する。
組織的なカーシェアリングサービス自体は1980年代後半にヨーロッパで始まり、日本では2002年に初の民営会社が発足したように決して新しいものではないが、近年のシェアリングエコノミー及びコネクティッドカーの普及により事業者数・利用者数ともに急激に規模を拡大しているサービスでもある。
現在のカーシェアリングサービスでは、利用者が利用開始時に自身のスマートフォン内のモバイルアプリケーションを用いて自動車を解錠したり、利用者の運転歴データを事業者が収集したりすることが一般的である。
例えば、パーク24グループが運営するカーシェアリングサービスであるタイムズカーにおいては、急加速・急減速などをリアルタイムで観測し、これらのデータを元に安全な運転を判定したうえで利用者にポイントを付与している。
無事故走行距離とポイントに応じて利用者は4段階に分かれたステージに振り分けられ、ステージに応じて割引などの恩恵を受けられる。

\section{本章のまとめ}
本章では、コネクティッドカーやテレマティクス、車両データを用いたサービスなど、本研究の背景について紹介した。
次章では、本研究の基盤となる技術について紹介する。