\chapter{結論}
\label{conclusion}

本章では、本研究のまとめと今後の課題を示す。

\section{本研究のまとめ}
本研究では、コネクティッドカーから抽出される運転歴データを対象とし、利用者自身がデータに対するアクセスおよびコントロール可能性を持ちながらも、その真正性を担保した形で第三者に提示可能とするモデルの構築を行なった。
従来のコネクティッドカーに関連する様々なサービスでは、自動車メーカーや特定の事業者に車両データが送られ、管理される構造が一般的であり、利用者は自身の運転データを自由に持ち出したり、異なるサービス間で再利用したりすることが困難であった。
一方で、データの真正性を担保するためには事業者の管理下にデータを置かざるを得ず、結果としてデータ構造のサイロ化やベンダーロックインが生じていた。
これらの課題に対し、本研究ではDIDsおよびVCsを用いることで、利用者を論理的にデータの流れの中心に位置させる新たな手法を提案した。
提案手法では、自動車が生成する運転データをもとにVCsを発行し、利用者がウォレットアプリケーションにこれを保持・提示する構成とすることで、データのコントロール可能性と真正性の両立を行なった。
さらに、提案手法に基づくシステムの設計を行い、自動車アプリケーション、モバイルアプリケーション、検証者アプリケーションから構成されるシステムを概念実証として実装した。
評価では、提案システムがデータの改ざん検知や発行元の検証を可能としつつ、利用者主体での運転歴データ提示を実現できることを確認した。
また、STRIDEによる脅威分析を通じて、想定される脅威とその影響が及びうる範囲を整理し、提案システムが一定のセキュリティ要件を満たし得ることも示した。
結果として、本研究では、コネクティッドカーにおける運転歴データの管理・提示に関して、利用者を中心として手法を提案し、その実現可能性を示した。
\section{本研究の課題と限界}
一方で、本研究にはいくつかの課題および限界が存在する。
第一に、本研究で実装したシステムは概念実証を目的としたものであり、実際の実運用を前提とした場合の性能の評価及び検証は十分に行えていない。
実際の実用環境においては、コネクティッドカーから抽出される膨大なデータの処理や、多数の事業者・利用者・検証者が同時に関与する状況が想定されるため、処理負荷や通信遅延に関する評価が今後の課題となる。
第二に、データの真正性は暗号学的手法により担保されるものの、自動車側で生成されるデータ自体の信頼性、すなわちセンサの故障や不正な改変をどこまで防げるかについては、本研究のスコープ外であり、完全な解決には至っていない。
車載システムの耐タンパ性やハードウェアセキュリティとの連携は、今後検討すべき重要な課題である。
第三に、Verifiable CredentialsおよびDecentralized Identifiersを前提としたエコシステムは、現時点では発展途上であり、標準仕様や実装の成熟度、事業者間の相互運用性といった点に課題が残る。
特に、自動車メーカーや保険会社、カーシェアリング事業者など、多様な主体が参加するためには、技術的側面だけでなく、制度面や運用ルールの整理も不可欠である。
今後の展望としては、実運用を想定した評価の実施に加え、選択的開示機能の高度化や、他分野との連携による応用可能性の検討が挙げられる。
これにより、運転歴データに限らず、モビリティ分野全体における利用者主導のデータ活用基盤へと発展させることが期待される。
%%% Local Variables:
%%% mode: japanese-latex
%%% TeX-master: "../thesis"
%%% End:
