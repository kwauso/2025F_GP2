\chapter{本研究における問題定義}
\label{issue}
本章では、既存システムにおいて本研究が着目する問題について述べる。

%%% Local Variables:
%%% mode: japanese-latex
%%% TeX-master: "./thesis"
%%% End:

\section{既存システムにおけるデータの流れ}
2.2節及び2.3節で紹介した既存サービスにおいて採用されているシステムにおいては、基本的にコネクティッドカーから抽出・収集されたデータは自動車から自動車メーカーやサービス事業者(以下、まとめて事業者と呼称)が接続しているクラウドへと流れ、クラウドで処理をされた後にそれぞれの事業者へと提供され蓄積されている。
システムにおけるデータの流れを見てみると、データは自動車と事業者間を繋ぐネットワーク内にのみ存在している(図~\ref{exsisting_dataflow}参照)。
トヨタ自動車のサービスであるT-Connectのように、モバイルアプリケーションへの配信などを通じて利用者にデータを提供するものも一部存在するが、あくまでサービスの一環としてわざわざデータをモバイルアプリケーションにも提供しているに過ぎず、本質的なデータの流れは変わらないと言える。
\begin{figure}[h]
    \centering
    \includegraphics[width=150mm]{src/public/dataflow.png}
    \caption{既存システムにおけるデータの流れ}\label{exsisting_dataflow}
\end{figure}
\section{既存システムの問題点}
前節で述べたデータの流れは、既存サービスが当てはまる登場人物が利用者と事業者のみ、というようないわば一対一のユースケースにおいては十分である。
しかしながら、特定の事業者との間でのみデータを共有する既存システムは、ベンダーロックインを容易に招き、利用者にサービスの乗り換えという選択肢を与えない。
また、実際のデータは利用者の自動車から事業者のクラウドに送信されて蓄積されているため、利用者が自身の自動車から抽出されたデータへのアクセスあるいはコントロールが可能であるとも言い難い。
現在では、事業者間でデータを共有できるかどうかはひとえにそれぞれの事業者の判断に委ねられている。
実際の例で言えば、タイムズカーにおいては、無事故走行距離とポイントに応じて利用者は4段階に分かれたステージに振り分けられ、ステージに応じて割引などの恩恵を受けられるが、このステージは他のカーシェアリング事業者(オリックスカーシェアや三井のカーシェア)においては全く意味をなさない。
言い換えるならば、複数の事業者にまたがってデータを共有する際、既存サービスでは事業者間の連携が不可欠である。
また、このようなシステムでは、事業者どうしが結託して利用者に断りなくデータを共有する事態が発生する可能性も否定できない。
\\
 次項からは、この問題を構成している要因についてそれぞれ述べる。
\subsection{データのサイロ化}
前述したように、既存サービスを構成しているシステムにおいてはコネクティッドカーから収集されたデータは自動車から事業者が接続しているクラウドへと流れ、クラウドで処理をされた後に事業者へと提供され蓄積される。
このようなシステムでは、自動車と事業者の確立するネットワークは非常に閉鎖的であり、常にその間にデータが閉じているためサイロ化とでも言うべき状態が出現する。
利用者が複数の事業者を利用する場合、互いに独立したサイロが事業者の数だけ存在していることになり、これが事業者間の連携を難しくしている要因の一つでもある。
\subsection{データに対するアクセス及びコントロールの欠如}
サイロが構成されていると言っても、データはコネクティッドカーから抽出された後に事業者と接続したクラウドにて処理・蓄積されるため、利用者が実際にデータにアクセスしたり、データをコントロールしたりすることは不可能に近い。
また、利用者は自身の自動車からどのようなデータが事業者に渡り、それがどのように扱われているかについて知ることはできない。
データが利用者のアクセスあるいはコントロール可能な場所にないため、事業者が仮にデータを不正に利用したり、許可なく他の事業者と共有したりしても利用者がそれを検知することができないという問題点も存在する。
仮に利用者が自分の自動車から抽出されたデータにアクセスし、それをコントロールすることが可能であれば、そのデータを複数の事業者に対して提示できる可能性が出てくるが、そのようなシステムは現在存在しない。
\subsection{データの真正性の担保}
仮に利用者のアクセスあるいはコントロール可能な場所にデータが移り、その上で利用者がデータを事業者に提示できるようになったとしても、提示された事業者がそのデータの真正性を検証できるとは限らない。
コネクティッドカーから直接データを抽出する構造のシステムでない以上、利用者がデータを偽造・改ざんしていたり、他人の自動車から抽出したデータを提示してきたりする可能性を否定できないためである。
そのため、既存サービスのシステムではデータの真正性を担保するために、事業者が認めた自動車メーカーが製造したコネクティッドカーから、安全な通信経路を経由してクラウドにデータを送信する構造を採用しているとも言える。
\subsection{各事業者のポリシーの相違}
さらに、これまでに述べた要因が取り除かれたとしても、各事業者のポリシーの相違によってデータを共有することができない場合がある。
これには2つのパターンがあり、データ取得時のポリシーと、データ利用時のポリシーの差異がある。
データ取得時のポリシーとは文字通りどのようなデータを取得するかを判断する際のポリシーであり、例えばA社ではスロットル開度を取得しない一方B社では取得しているとしたら、A社からB社へデータを持ち越すのは困難になる。
データ利用時のポリシーとは取得したデータを用いて利用者を判断する際のポリシーであり、先ほどの例ではタイムズカーではこれに基づき利用者を4段階のステージに振り分けるが、ポリシーを共有しない他社ではこのステージでは全く意味がなくなってしまう。
このような場合はポリシー実行後の状態のみを共有するのではなく、元のデータとポリシーの両方を合わせて提示するなどの工夫が必要になるだろう。
いずれにせよ、各事業者はそれぞれ異なるポリシーを持っており、判断に用いられるデータは同じでも、例えば保険会社では総合した安全運転の実績を重視する一方、カーシェアリングサービスでは急な加減速の有無を重視するなど、ポリシーによってデータの評価基準が異なってしまい、複数の事業者にまたがってデータを共有することを難しくしている。
\section{本章のまとめ}
本章では、既存サービスを構成しているサービスにおけるデータの流れを示し、その問題点を洗い出した。
次章では、以上の問題点を解決するシステムを提案する。