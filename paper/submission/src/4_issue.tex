\chapter{本研究における問題定義}
\label{issue}
本章では、既存システムにおいて本研究が着目する問題について述べる。

%%% Local Variables:
%%% mode: japanese-latex
%%% TeX-master: "./thesis"
%%% End:

\section{既存システムにおけるデータの流れ}
2.2節及び2.3節で紹介した既存サービスにおいて用いられているシステムでは、基本的にコネクティッドカーから収集されたデータは自動車からサービスの事業者が接続しているクラウドへと流れ、クラウドで処理をされた後にサービス事業者・メーカー事業者へと提供されている。
システムのデータの流れを見てみると、基本的にデータは自動車と事業者間を繋ぐネットワーク内にのみ存在していると言える。
トヨタの提供するT-Connectのように、モバイルアプリケーションへの配信などを通じて利用者にデータを提供するサービスも一部存在するが、あくまでサービスに一環としてわざわざデータをモバイルアプリケーションにも提供しているに過ぎず、本質的なデータの流れは変わらないと言える。
\section{既存システムの問題点}
前節で述べたデータの流れは、既存サービスのような登場人物が利用者と事業者のみ、というような一対一のユースケースにおいては十分である。
しかしながら、特定の事業者とのみデータを共有する既存システムは、ベンダーロックインを容易に招き、利用者にサービスの乗り換えという選択肢を与えない。
また、データを共有と言えど、実際のデータは利用者の自動車から事業者のクラウドに蓄積されているため、利用者が自身の自動車から抽出したデータへのアクセスが可能であるとも言い難い。
例えば、利用者が保険会社を乗り換えても利用ベース保険の実績を引き継ぐことができるかどうかや、カーシェアリングサービスをまたいでも安全運転の履歴により割引を受けられるかどうかはひとえに保険会社やカーシェアリングサービスの判断に委ねられている。
実際に例で言うならば、タイムズカーにおいては、無事故走行距離とポイントに応じて利用者は4段階に分かれたステージに振り分けられ、ステージに応じて割引などの恩恵を受けられるが、このステージは他のカーシェアリング事業者(オリックスカーシェアや三井のカーシェア)においては全く意味をなさない。
言い換えるならば、複数の事業者にまたがってデータを共有する際、既存サービスでは事業者やメーカー間の連携が不可欠であるという問題点が存在する。
また、このようなシステムでは、複数の事業者が結託して利用者に断りなくデータを共有する事態が発生する可能性も否定できない。
\begin{figure}[h]
    \centering
    \includegraphics[width=150mm]{src/public/dataflow.png}
    \caption{既存システムにおけるデータの流れ}
\end{figure}
\\
 次項からは、この問題を構成している要因についてそれぞれ述べる。
\subsection{データのサイロ化}
前述したように、既存サービスを構成しているシステムにおいてはコネクティッドカーから収集されたデータは自動車から事業者が接続しているクラウドへと流れ、クラウドで処理をされた後に事業者へと提供される。
このようなシステムでは、自動車と事業者の共有するネットワークは非常に閉鎖的であり、常にその間にデータが閉じているためサイロ化とでも言うべき状態が発生する。
利用者が複数事業者を利用する場合、互いに独立したサイロが事業者の数だけ複数存在していることになり、これが複数の事業者間の連携を難しくしている要因の一つである。
\subsection{データのコントロール権の欠如}
また、サイロが構成されていると言っても、データはコネクティッドカーから抽出された後に事業者と接続したクラウドにて処理・蓄積されるため、利用者が実際にデータにアクセスしたり、データをコントロールしたりすることは不可能に近い。
さらに言うならば、利用者は自身の自動車からどのようなデータが事業者に渡り、それがどのように扱われているかについて知ることはできない。
データが利用者のアクセス・コントロール可能な場所にないため、事業者が仮にデータを不正に利用したり、転売したとしても利用者がそれを検知することができないという問題点も存在する。
仮に利用者が自分の自動車から抽出されたデータにアクセスし、それをコントロールすることが可能であれば、そのデータを複数の事業者に対して提示できる可能性が出てくるが、そのようなシステムは現在存在しない。
\subsection{データの真正性の担保}
仮に利用者のアクセス・コントローラ可能な場所にデータが移り、その上で利用者がデータを提示してきたとしても、それを事業者が真正であると検証できるとは限らない。
自動車から直にデータを抽出するアーキテクチャでない以上、利用者がデータを偽造・改ざんしていたり、他人の自動車から抽出したデータを提示してきたりする可能性を否定できないからである。
そのため、既存システムではデータの真正性を担保するために、事業者が認めた自動車メーカーが製造したコネクティッドカーから安全な通信経路を経由してクラウドにデータを送信しているとも言える。
\subsection{各事業者のポリシーの相違}
さらに、これまでに述べた要因が取り除かれたとしても、各事業者のポリシーの相違によってデータを共有することができない場合がある。
これには2つのパターンがあり、データ取得時のポリシーと、データ利用時のポリシーの差異がある。
データ取得時のポリシーとは文字通りどのようなデータを取得するかを判断する際のポリシーであり、例えばA社ではスロットル開度を取得しない一方B社では取得しているとしたら、A社からB社へデータを持ち越すのは困難になる。
データ利用時のポリシーとは取得したデータを用いて利用者を判断する際のポリシーであり、先ほどの例ではタイムズカーではこれに基づき利用者を4段階のステージに振り分けるが、ポリシーを共有しない他社ではこのステージでは全く意味がなくなってしまう。
このような場合はポリシー実行後の状態のみを共有するのではなく、元のデータとポリシーの両方を合わせて提示するなどの工夫が必要になるだろう。
いずれにせよ、各事業者はそれぞれ異なるポリシーを持っており、判断に用いられるデータは同じでも、例えば保険会社では総合した安全運転の実績を重視する一方、カーシェアリングサービスでは急な加減速の有無を重視するなど、ポリシーによってデータの評価基準が異なってしまい、複数の事業者にまたがってデータを共有することを難しくしている。
\section{本章のまとめ}
本章では、既存サービスを構成しているサービスにおけるデータの流れを示し、その問題点を洗い出した。
次章では、以上の問題点を解決するシステムを提案する。