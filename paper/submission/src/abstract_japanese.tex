卒業論文要旨 - 2025年度 (令和7年度)
\begin{center}
\begin{large}
\begin{tabular}{|M{0.97\linewidth}|}
    \hline
      \title \\
    \hline
\end{tabular}
\end{large}
\end{center}

~ \\

コネクテッドカーの普及に伴い、テレマティクスを通じて運転データが抽出・収集され、利用ベース保険やカーシェアリングなど、さまざまなサービスで活用されている。
しかし、既存のシステムでは、運転データは自動車メーカーやサービス事業者の提供する閉じたインフラ内で保存・管理されるのが一般的であり、その結果、データのサイロ化やベンダーロックインが生じ、利用者が自身の自動車から抽出したデータにアクセスあるいはコントロールできないという課題がある。
一方で、サービス事業者にとっては、運転データの真正性が担保されることは依然として重要である。
\\
 そこで本研究では、自動車とサービス事業者の間に利用者を論理的に位置付ける、新たなコネクティッドカー向けの新たな運転データの管理・提示モデルを提案する。
提案モデルでは、暗号学的な技術を用いることでデータの真正性を担保しつつ、利用者が自身の運転データへアクセスおよびコントロールできることを可能にする。
このモデルを実現するために、Verifiable Credentials および Decentralized Identifiers に基づくシステムを設計し、運転データを複数の主体間で発行、保持、検証できる仕組みを構築した。
\\
 さらに、概念実証を行うために提案手法に基づいたシステムを実装した。
実装した後には、本システムについて、データに対するアクセスおよびコントロール可能性、データの真正性の担保の観点から定性的評価を行うとともに、概念実証から検証できる実現可能性についても評価し、最後にSTRIDEに基づく脅威分析を実施した。
その結果、提案手法は、データに対するアクセスおよびコントロール可能性とデータの真正性の担保を両立した上で実現可能性があることを示した。
\\
 本研究の貢献により、特定のベンダーに依存しない開かれた自動車データエコシステムが構築可能になることが期待できる。

~ \\
キーワード:\\
\underline{1. コネクティッドカー},
\underline{2. Verifiable Credentials}
\begin{flushright}
\dept \\
\author
\end{flushright}
