\appendix
\chapter{付録}

\section{実装のコードスニペット}
以下は実装のコードスニペットである。
\begin{mylisting}[language=c++,caption=ハッシュチェーンの生成アルゴリズム]
hash_i = SHA256(
    prevHash (32 bytes) ||
    timestamp (8 bytes, big-endian double) ||
    speed (8 bytes, big-endian double) ||
    acceleration (8 bytes, big-endian double) ||
    yawRate (8 bytes, big-endian double) ||
    steeringAngle (8 bytes, big-endian double)
)
\end{mylisting}
\begin{mylisting}[language=c++,caption=セグメントの型定義]
interface Segment {
    segmentStartTime: string;      // タイムスタンプ
    durationSec: number;
    aggregatedMetrics: {
        speed: { avg, max, min },
        acceleration: { avg, max, min },
        yawRate: { avg, max, min },
        steeringAngle: { avg, max, min }
    };
    hashChain: {
        start: string;  // セグメントの最初のハッシュ
        end: string;    // セグメントの最後のハッシュ
    };
}
\end{mylisting}
\begin{mylisting}[language=c++,caption=発行されるVCsの型定義]
interface DrivingEvaluationCredential {
    id?: string;
    "@context": string[];
    type: string[];
    issuer: string;              // 発行者のDIDs
    issuanceDate: string;
    credentialSubject: {
        id?: string;               // 被発行者のDIDs
        segmentStartTime: string;
        durationSec: number;
        aggregatedMetrics: AggregatedMetrics;
        hashChain: HashChain;
    };
    proof: {
        type: string;
        created: string;
        verificationMethod: string;
        proofPurpose: string;
        jws?: string;
        proofValue?: string;
    };
    }
\end{mylisting}
\begin{mylisting}[language=c++,caption=発行されるVPの型定義]
interface VerifiablePresentation {
    "@context": string[];
    type: string[];
    holder: string;                    // HolderのDIDs
    verifiableCredential: DrivingEvaluationCredential[];  // 含まれるVCs
    proof: {
        type: string;
        created: string;
        verificationMethod: string;
        proofPurpose: string;
        jws?: string;
        proofValue?: string;
    };
    }
\end{mylisting}
\section{STRIDEによる脅威分析の結果}
第7章でSTRIDEによる脅威分析を行った。
以下はその結果を信頼境界ごとに示した表である。
なお、各脅威の番号は便宜上振り分けたものであり、別段の意味はない。
\begin{table}[h]
    \centering
    \caption{自動車における脅威}
    \label{tab:stride_car}
    \begin{tabular}{llllp{5.5cm}}
        \hline
        \hline
        番号 & 要素 & 要素の種類 & 脅威のカテゴリ & 脅威の概要 \\
        \hline
        1 & VCs発行状態 & データストア & 改ざん & VCsの発行状態が改ざんされる \\
        2 & DIDsストア & データストア & 改ざん & DIDsが改ざんされる \\
        3 & 鍵ストア & データストア & 改ざん & 鍵情報が改ざんされる \\
        4 & 連携状態 & データストア & 改ざん & モバイルとの連携状態が改ざんされる \\
        5 & VCs発行プロセス & プロセス & なりすまし & 第三者がVCsの発行を受ける \\
        6 & 連携プロセス & プロセス & なりすまし & 第三者が連携を試みてくる \\
        7 & VCs発行プロセス & プロセス & 改ざん & VCsの発行プロセスが不正に行われる \\
        8 & 連携プロセス & プロセス & 改ざん & モバイルとの連携プロセスが不正に行われる \\
        9 & DIDs & データフロー & 改ざん & DIDsが改ざんされる \\
        10 & 鍵ペア & データフロー & 改ざん & 鍵情報が改ざんされる \\
        11 & 発行状態 & データフロー & 改ざん & 発行状態が改竄される \\
        12 & 連携状態 & データフロー & 改ざん & 連携状態が改ざんされる \\
        13 & 公開鍵情報 & データフロー & 改ざん & 公開鍵情報が改ざんされる \\
        \hline
    \end{tabular}
\end{table}
\begin{table}[h]
    \centering
    \caption{モバイルにおける脅威}
    \label{tab:stride_mobile}
    \begin{tabular}{llllp{5.5cm}}
        \hline
        \hline
        番号 & 要素 & 要素の種類 & 脅威のカテゴリ & 脅威の概要 \\
        \hline
        1 & 連携状態 & データストア & 改ざん & 自動車との連携状態が改ざんされる \\
        2 & DIDsストア & データストア & 改ざん & DIDsが改ざんされる \\
        3 & 鍵ストア & データストア & 改ざん & 鍵情報が改ざんされる \\
        4 & VCs送受信プロセス & プロセス & なりすまし & 第三者がVCsの受信を受ける \\
        5 & 連携プロセス & プロセス & なりすまし & 第三者が連携を試みてくる \\
        6 & VCs発行プロセス & プロセス & 改ざん & VCsの発行プロセスが不正に行われる \\
        7 & 連携プロセス & プロセス & 改ざん & モバイルとの連携プロセスが不正に行われる \\
        8 & 連携アプリ & 外部の主体 & 改ざん & 連携アプリケーションが改ざんされる \\
        9 & ウォレット & 外部の主体 & 改ざん & ウォレットアプリケーションが改ざんされる \\
        10 & DIDs & データフロー & 改ざん & DIDsが改ざんされる \\
        11 & 鍵ペア & データフロー & 改ざん & 鍵情報が改ざんされる \\
        12 & VCs & データフロー & 改ざん & VCsが改竄される \\
        13 & 連携状態 & データフロー & 改ざん & 連携状態が改ざんされる \\
        14 & 公開鍵情報 & データフロー & 改ざん & 公開鍵情報が改ざんされる \\
        \hline
    \end{tabular}
\end{table}
\begin{table}[h]
    \centering
    \caption{検証者における脅威}
    \label{tab:stride_verifier}
    \begin{tabular}{llllp{5.5cm}}
        \hline
        \hline
        番号 & 要素 & 要素の種類 & 脅威のカテゴリ & 脅威の概要 \\
        \hline
        1 & 外部VDR & データストア & 改ざん & 自動車や利用者の公開鍵情報が改ざんされる \\
        2 & 検証者 & 外部の主体 & なりすまし & 第三者が検証者になりすます \\
        3 & 鍵ストア & データストア & 改ざん & 鍵情報が改ざんされる \\
        4 & VP構成プロセス & プロセス & 改ざん & 悪意ある形でVPを発行する \\
        5 & VP & データフロー & 改ざん & VPが改竄される \\
        6 & 公開鍵情報 & データフロー & 改ざん & 公開鍵情報が改ざんされる \\
        \hline
    \end{tabular}
\end{table}