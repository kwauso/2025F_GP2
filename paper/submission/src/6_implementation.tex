\chapter{実装}
\label{implementation}
本章では、第5章で提案したシステムの概念実証としての実装について述べる。

%%% Local Variables:
%%% mode: japanese-latex
%%% TeX-master: "../bthesis"
%%% End:

\section{実装の概要}
前章で提案した手法を実装し、実際に動作するシステムを構築する。
実装するのは自動車アプリケーション、モバイルアプリケーション、検証者アプリケーションの3つであるが、自動車アプリケーションに関しては実際の自動車上で動作する車載アプリケーションを開発することは大きな労力を要するため、車載アプリケーションをエミュレートする形で実装した。
また、実際に自動車からデータを抽出するのではなく実際のデータを模したテストデータを利用した。
なお、この実装は提案手法の概念実証として行い、提案するシステムが十分に実現可能であるかどうかを検証することを目的としているため、提案システムのコアの部分、すなわちデータの真正性の担保とコントロール可能性の両立を実現する役割を持つコンポーネントを主に実装している。
\subsection{機能要件}
実装にあたり、それぞれのアプリケーションの機能要件については、以下とする。
\begin{enumerate}
    \item 自動車アプリケーション \newline
        ・基盤内の鍵ペアを用い、モバイルアプリケーションとの間の信頼できる通信経路の確立 \newline
        ・自動車内の各センサから取得した情報をデータ辞書に基づいてVCに加工し、モバイルアプリケーションへ送信
    \item モバイルアプリケーション \newline
        ・自動車との間の信頼できる通信経路の確立 \newline
        ・自動車から送信されたVCをウォレットアプリケーションに蓄積し、VPを構成 \newline
        ・VPを検証者に提示
    \item 検証者アプリケーション \newline
        ・受信したVPを確認してその有効性及び真正性を検証し、データを取り出す
\end{enumerate}
\begin{description}
    \item[\textbf{自動車アプリケーション}]
    テストデータを基にVCsを発行し、VCを発行する
    \item[\textbf{モバイルアプリケーション}]
    VCsを受け取って蓄積し、VPを構成した上で提示する
    \item[\textbf{検証者アプリケーション}]
    提示されたVPからDIDsを取得した上でDID Documentを解決し、VPを検証する
\end{description}
\subsection{システム構成}
全体のシステム構成は以下である。
\subsection{全体の開発環境}
実装にあたり、開発環境は以下を用いた。
\section{データモデル}
データモデルは第5章で提案したものを用いる。
\subsection{実装に用いたテストデータ}
\section{自動車アプリ}
実装にあたり、自動車アプリケーションはRaspberry Piを用いたエミュレータとして実装した。
\section{モバイルアプリ}
実装にあたり、モバイルアプリケーションはFlutterを用いて実装した。
\section{検証者アプリ}
実装にあたり、検証者アプリケーションはTypescriptを用いて実装した。
\section{VC生成・署名の方式}
\section{本章のまとめ}
