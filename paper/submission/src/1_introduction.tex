\chapter{序論}
\label{introduction}

本章では本研究の背景、課題及び手法を提示し、本研究の概要を示す。

\section{はじめに}

インターネットの発展に伴い、あらゆるモノがネットワークに接続されることはもはや自然なことになった。
Internet of Thingsと呼ばれるこの概念は急速に普及しており、ネットワークと接続するモノの範囲は拡大する一方である。
自動車もその例外ではなく、車両に通信機能を搭載することで、外部ネットワークと常時接続される「コネクティッドカー」へと進化してきた~\cite{sato2009internetcar}。
とりわけ、テレマティクスと呼ばれる、自動車をはじめとする移動体に通信システムを搭載し、リアルタイムでデータを外部とやり取りすることを可能にする技術の発展は、自動車の運転や利用状況に関する詳細なデータ収集を可能にしてきた。
これらのデータは長らく自動車の安全運行のためのシステム、例えば緊急通報システムやカーナビゲーションシステムといった形で交通安全に寄与してきたが、現代ではこれらのデータが非常に個人的なものであることに着目し、自動車の利用者一人ひとりに最適化したサービスを提供しようとする試みも行われている。
利用ベース保険やカーシェアリングサービスはその代表であり、車両から抽出したデータに基づく多様な価値提供が進んでいる。
\\
 しかしながら、現在普及しつつあるコネクティッドカーを基盤としたサービスにおいては、自動車から抽出されたデータは自動車メーカーや特定のサービス事業者のクラウドやデータセンター内に閉じた形で保管されていることがほとんどである。
そのため、利用者が自身の運転データを外部へ持ち出したり、サービスを乗り換える際に過去の運転データを引き継いだりすることが非常に困難である、すなわちベンダーロックインに陥るという問題が発生している。
また、データが分断されてサイロ構造に閉じ込められていると、利用者自身が自らの自動車から抽出されたデータへアクセスあるいはコントロールできないという問題も発生する。
一方で、事業者がこのような構造を取らざるを得ない理由の一つとして、自動車から直接データを取得しない限り、データの真正性を担保できないという技術的な問題も存在する。
\\
 そこで本研究では、このような問題に対し、利用者を自動車と事業者の中間に論理的に配置してデータへのアクセスおよびコントロール可能性を持たせ、その上でデータの真正性を担保する新たな運転歴データの管理・提示モデルを提案する。
さらにこのモデルを評価するために、Verifiable CredentialsとDecentralized Identifiersという技術を用いたシステムを設計した上で概念実証として実装し、実際に動作確認を行う。
以上により、利用者自身がデータに対するアクセスおよびコントロールが可能でありながら、データの真正性が担保され、複数の主体間で安全かつ拡張可能な形で運転歴データの利用が可能となる仕組みを構築する。
\section{本論文の構成}

本論文における以降の構成は次の通りである.

~\ref{background}章では、本研究の背景を述べる。
~\ref{technology}章では,本研究において基盤となる技術を述べる.
~\ref{issue}章では、本研究における問題の定義と、解決するための要件の整理を行う。
~\ref{proposed}章では、本研究の提案手法を述べる。
~\ref{implementation}章では、~\ref{proposed}章で述べたシステムの実装について述べる。
~\ref{evaluation}章では、\ref{issue}章で求められた課題に対しての評価を行い、考察する。
~\ref{conclusion}章では、本研究のまとめと課題、そして今後の展望についてまとめる。


%%% Local Variables:
%%% mode: japanese-latex
%%% TeX-master: "../thesis"
%%% End:
